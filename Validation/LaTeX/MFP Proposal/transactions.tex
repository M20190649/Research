\section*{MP finding per window...}
	\vspace{1cm}
	\psscalebox{1.0 1.0} {
	\begin{pspicture}(0,0.0)(13,12)
				\psline[linecolor=black, linewidth=0.05] (0,-1)(0,12) \uput[90](0, 12){$t_{0}$}
		\psline[linecolor=black, linewidth=0.02] (1,-1)(1,12) \uput[90](1, 12){$t_{1}$}
		\psline[linecolor=black, linewidth=0.02] (2,-1)(2,12) \uput[90](2, 12){$t_{2}$}
		\psline[linecolor=black, linewidth=0.02] (3,-1)(3,12) \uput[90](3, 12){$t_{3}$}
		\psline[linecolor=black, linewidth=0.05] (4,-1)(4,12) \uput[90](4, 12){$t_{4}$}

		\uput[180](0,0){\textcolor{blue}{$A$}}
		\uput[180](0,1){\textcolor{red}{$B$}}
		\uput[180](0,2){\textcolor{cyan}{$C$}}
		\uput[180](0,3){\textcolor{magenta}{$D$}}
		\uput[180](0,4){\textcolor{orange}{$E$}}
		\uput[180](0,5){\textcolor{green}{$F$}}
		\uput[180](0,6){\textcolor{gray}{$G$}}
		\uput[180](0,7){\textcolor{violet}{$H$}}
		\uput[180](0,8){\textcolor{olive}{$I$}}
		\uput[180](0,9){\textcolor{brown}{$J$}}
		\uput[180](0,10){\textcolor{purple}{$K$}}
		\uput[180](0,11){\textcolor{black}{$L$}}

		\dotnode[dotstyle=*,linecolor=red,dotsize=0.2]     (0.0,1.24)  {B 0}
		\dotnode[dotstyle=*,linecolor=cyan,dotsize=0.2]    (0.0,1.8)   {C 0}
		\dotnode[dotstyle=*,linecolor=magenta,dotsize=0.2] (0.0,2.89)  {D 0}
		\dotnode[dotstyle=*,linecolor=orange,dotsize=0.2]  (0.0,3.9)   {E 0}
		\dotnode[dotstyle=*,linecolor=gray,dotsize=0.2]    (0.0,5.96)  {G 0}
		\dotnode[dotstyle=*,linecolor=olive,dotsize=0.2]   (0.0,8.32)  {I 0}
		\dotnode[dotstyle=*,linecolor=brown,dotsize=0.2]   (0.0,9.39)  {J 0}
		\dotnode[dotstyle=*,linecolor=purple,dotsize=0.2]  (0.0,9.95)  {K 0}
		
		\dotnode[dotstyle=*,linecolor=blue,dotsize=0.2]    (1.0,0.28)  {A 1}
		\dotnode[dotstyle=*,linecolor=red,dotsize=0.2]     (1.0,1.45)  {B 1}
		\dotnode[dotstyle=*,linecolor=cyan,dotsize=0.2]    (1.0,2.42)  {C 1}
		\dotnode[dotstyle=*,linecolor=magenta,dotsize=0.2] (1.0,2.93)  {D 1}
		\dotnode[dotstyle=*,linecolor=green,dotsize=0.2]   (1.0,5.13)  {F 1}
		\dotnode[dotstyle=*,linecolor=violet,dotsize=0.2]  (1.0,7.41)  {H 1}
		\dotnode[dotstyle=*,linecolor=olive,dotsize=0.2]   (1.0,8.31)  {I 1}
		\dotnode[dotstyle=*,linecolor=brown,dotsize=0.2]   (1.0,8.68)  {J 1}
		\dotnode[dotstyle=*,linecolor=black,dotsize=0.2]   (1.0,10.68) {L 1}

		\dotnode[dotstyle=*,linecolor=blue,dotsize=0.2]   (2.0,-0.14) {A 2}
		\dotnode[dotstyle=*,linecolor=cyan,dotsize=0.2]   (2.0,2.17)  {C 2}
		\dotnode[dotstyle=*,linecolor=orange,dotsize=0.2] (2.0,4.09)  {E 2}
		\dotnode[dotstyle=*,linecolor=green,dotsize=0.2]  (2.0,4.57)  {F 2}
		\dotnode[dotstyle=*,linecolor=gray,dotsize=0.2]   (2.0,5.51)  {G 2}
		\dotnode[dotstyle=*,linecolor=brown,dotsize=0.2]  (2.0,9.16)  {J 2}
		\dotnode[dotstyle=*,linecolor=purple,dotsize=0.2] (2.0,10.0)  {K 2}
		\dotnode[dotstyle=*,linecolor=black,dotsize=0.2]  (2.0,10.57) {L 2}

		\dotnode[dotstyle=*,linecolor=blue,dotsize=0.2]   (3.0,0.01)  {A 3}
		\dotnode[dotstyle=*,linecolor=red,dotsize=0.2]    (3.0,1.15)  {B 3}
		\dotnode[dotstyle=*,linecolor=cyan,dotsize=0.2]   (3.0,1.76)  {C 3}
		\dotnode[dotstyle=*,linecolor=orange,dotsize=0.2] (3.0,4.09)  {E 3}
		\dotnode[dotstyle=*,linecolor=gray,dotsize=0.2]   (3.0,5.72)  {G 3}
		\dotnode[dotstyle=*,linecolor=violet,dotsize=0.2] (3.0,6.94)  {H 3}
		\dotnode[dotstyle=*,linecolor=olive,dotsize=0.2]  (3.0,7.53)  {I 3}
		\dotnode[dotstyle=*,linecolor=brown,dotsize=0.2]  (3.0,9.5)   {J 3}
		\dotnode[dotstyle=*,linecolor=black,dotsize=0.2]  (3.0,10.91) {L 3}

		\dotnode[dotstyle=*,linecolor=blue,dotsize=0.2]    (4.0,0.35) {A 4}
		\dotnode[dotstyle=*,linecolor=cyan,dotsize=0.2]    (4.0,1.79) {C 4}
		\dotnode[dotstyle=*,linecolor=magenta,dotsize=0.2] (4.0,2.52) {D 4}
		\dotnode[dotstyle=*,linecolor=orange,dotsize=0.2]  (4.0,4.33) {E 4}
		\dotnode[dotstyle=*,linecolor=green,dotsize=0.2]   (4.0,5.15) {F 4}
		\dotnode[dotstyle=*,linecolor=gray,dotsize=0.2]    (4.0,6.21) {G 4}
		\dotnode[dotstyle=*,linecolor=violet,dotsize=0.2]  (4.0,6.57) {H 4}
		\dotnode[dotstyle=*,linecolor=olive,dotsize=0.2]   (4.0,7.99) {I 4}
		\dotnode[dotstyle=*,linecolor=brown,dotsize=0.2]   (4.0,8.98) {J 4}
		\dotnode[dotstyle=*,linecolor=black,dotsize=0.2]  (4.0,10.31) {L 4}

		\ncline[linecolor=blue,linestyle=dashed]{A 1}{A 2}
		\ncline[linecolor=blue,linestyle=dashed]{A 2}{A 3}
		\ncline[linecolor=blue,linestyle=dashed]{A 3}{A 4}

		\ncline[linecolor=red,linestyle=dashed]{B 0}{B 1}
		\ncline[linecolor=red,linestyle=dashed]{B 3}{B 3}

		\ncline[linecolor=cyan,linestyle=dashed]{C 0}{C 1}
		\ncline[linecolor=cyan,linestyle=dashed]{C 1}{C 2}
		\ncline[linecolor=cyan,linestyle=dashed]{C 2}{C 3}
		\ncline[linecolor=cyan,linestyle=dashed]{C 3}{C 4}

		\ncline[linecolor=magenta,linestyle=dashed]{D 0}{D 1}

		\ncline[linecolor=orange,linestyle=dashed]{E 2}{E 3}
		\ncline[linecolor=orange,linestyle=dashed]{E 3}{E 4}

		\ncline[linecolor=green,linestyle=dashed]{F 1}{F 2}

		\ncline[linecolor=gray,linestyle=dashed]{G 0}{G 0}
		\ncline[linecolor=gray,linestyle=dashed]{G 2}{G 3}
		\ncline[linecolor=gray,linestyle=dashed]{G 3}{G 4}

		\ncline[linecolor=violet,linestyle=dashed]{H 1}{H 1}
		\ncline[linecolor=violet,linestyle=dashed]{H 3}{H 4}

		\ncline[linecolor=olive,linestyle=dashed]{I 0}{I 1}
		\ncline[linecolor=olive,linestyle=dashed]{I 3}{I 4}

		\ncline[linecolor=brown,linestyle=dashed]{J 0}{J 1}
		\ncline[linecolor=brown,linestyle=dashed]{J 1}{J 2}
		\ncline[linecolor=brown,linestyle=dashed]{J 2}{J 3}
		\ncline[linecolor=brown,linestyle=dashed]{J 3}{J 4}

		\ncline[linecolor=black,linestyle=dashed]{L 1}{L 2}
		\ncline[linecolor=black,linestyle=dashed]{L 2}{L 3}
		\ncline[linecolor=black,linestyle=dashed]{L 3}{L 4}

	\end{pspicture}
}
\clearpage
\section*{MP finding per window...}
	\vspace{1cm}
	\psscalebox{1.0 1.0} {
	\begin{pspicture}(0,0.0)(13,12)
				\psline[linecolor=black, linewidth=0.05] (0,-1)(0,12) \uput[90](0, 12){$t_{0}$}
		\psline[linecolor=black, linewidth=0.02] (1,-1)(1,12) \uput[90](1, 12){$t_{1}$}
		\psline[linecolor=black, linewidth=0.02] (2,-1)(2,12) \uput[90](2, 12){$t_{2}$}
		\psline[linecolor=black, linewidth=0.02] (3,-1)(3,12) \uput[90](3, 12){$t_{3}$}
		\psline[linecolor=black, linewidth=0.05] (4,-1)(4,12) \uput[90](4, 12){$t_{4}$}

		\uput[180](0,0){\textcolor{blue}{$A$}}
		\uput[180](0,1){\textcolor{red}{$B$}}
		\uput[180](0,2){\textcolor{cyan}{$C$}}
		\uput[180](0,3){\textcolor{magenta}{$D$}}
		\uput[180](0,4){\textcolor{orange}{$E$}}
		\uput[180](0,5){\textcolor{green}{$F$}}
		\uput[180](0,6){\textcolor{gray}{$G$}}
		\uput[180](0,7){\textcolor{violet}{$H$}}
		\uput[180](0,8){\textcolor{olive}{$I$}}
		\uput[180](0,9){\textcolor{brown}{$J$}}
		\uput[180](0,10){\textcolor{purple}{$K$}}
		\uput[180](0,11){\textcolor{black}{$L$}}

		\dotnode[dotstyle=*,linecolor=red,dotsize=0.2]     (0.0,1.24)  {B 0}
		\dotnode[dotstyle=*,linecolor=cyan,dotsize=0.2]    (0.0,1.8)   {C 0}
		\dotnode[dotstyle=*,linecolor=magenta,dotsize=0.2] (0.0,2.89)  {D 0}
		\dotnode[dotstyle=*,linecolor=orange,dotsize=0.2]  (0.0,3.9)   {E 0}
		\dotnode[dotstyle=*,linecolor=gray,dotsize=0.2]    (0.0,5.96)  {G 0}
		\dotnode[dotstyle=*,linecolor=olive,dotsize=0.2]   (0.0,8.32)  {I 0}
		\dotnode[dotstyle=*,linecolor=brown,dotsize=0.2]   (0.0,9.39)  {J 0}
		\dotnode[dotstyle=*,linecolor=purple,dotsize=0.2]  (0.0,9.95)  {K 0}
		
		\dotnode[dotstyle=*,linecolor=blue,dotsize=0.2]    (1.0,0.28)  {A 1}
		\dotnode[dotstyle=*,linecolor=red,dotsize=0.2]     (1.0,1.45)  {B 1}
		\dotnode[dotstyle=*,linecolor=cyan,dotsize=0.2]    (1.0,2.42)  {C 1}
		\dotnode[dotstyle=*,linecolor=magenta,dotsize=0.2] (1.0,2.93)  {D 1}
		\dotnode[dotstyle=*,linecolor=green,dotsize=0.2]   (1.0,5.13)  {F 1}
		\dotnode[dotstyle=*,linecolor=violet,dotsize=0.2]  (1.0,7.41)  {H 1}
		\dotnode[dotstyle=*,linecolor=olive,dotsize=0.2]   (1.0,8.31)  {I 1}
		\dotnode[dotstyle=*,linecolor=brown,dotsize=0.2]   (1.0,8.68)  {J 1}
		\dotnode[dotstyle=*,linecolor=black,dotsize=0.2]   (1.0,10.68) {L 1}

		\dotnode[dotstyle=*,linecolor=blue,dotsize=0.2]   (2.0,-0.14) {A 2}
		\dotnode[dotstyle=*,linecolor=cyan,dotsize=0.2]   (2.0,2.17)  {C 2}
		\dotnode[dotstyle=*,linecolor=orange,dotsize=0.2] (2.0,4.09)  {E 2}
		\dotnode[dotstyle=*,linecolor=green,dotsize=0.2]  (2.0,4.57)  {F 2}
		\dotnode[dotstyle=*,linecolor=gray,dotsize=0.2]   (2.0,5.51)  {G 2}
		\dotnode[dotstyle=*,linecolor=brown,dotsize=0.2]  (2.0,9.16)  {J 2}
		\dotnode[dotstyle=*,linecolor=purple,dotsize=0.2] (2.0,10.0)  {K 2}
		\dotnode[dotstyle=*,linecolor=black,dotsize=0.2]  (2.0,10.57) {L 2}

		\dotnode[dotstyle=*,linecolor=blue,dotsize=0.2]   (3.0,0.01)  {A 3}
		\dotnode[dotstyle=*,linecolor=red,dotsize=0.2]    (3.0,1.15)  {B 3}
		\dotnode[dotstyle=*,linecolor=cyan,dotsize=0.2]   (3.0,1.76)  {C 3}
		\dotnode[dotstyle=*,linecolor=orange,dotsize=0.2] (3.0,4.09)  {E 3}
		\dotnode[dotstyle=*,linecolor=gray,dotsize=0.2]   (3.0,5.72)  {G 3}
		\dotnode[dotstyle=*,linecolor=violet,dotsize=0.2] (3.0,6.94)  {H 3}
		\dotnode[dotstyle=*,linecolor=olive,dotsize=0.2]  (3.0,7.53)  {I 3}
		\dotnode[dotstyle=*,linecolor=brown,dotsize=0.2]  (3.0,9.5)   {J 3}
		\dotnode[dotstyle=*,linecolor=black,dotsize=0.2]  (3.0,10.91) {L 3}

		\dotnode[dotstyle=*,linecolor=blue,dotsize=0.2]    (4.0,0.35) {A 4}
		\dotnode[dotstyle=*,linecolor=cyan,dotsize=0.2]    (4.0,1.79) {C 4}
		\dotnode[dotstyle=*,linecolor=magenta,dotsize=0.2] (4.0,2.52) {D 4}
		\dotnode[dotstyle=*,linecolor=orange,dotsize=0.2]  (4.0,4.33) {E 4}
		\dotnode[dotstyle=*,linecolor=green,dotsize=0.2]   (4.0,5.15) {F 4}
		\dotnode[dotstyle=*,linecolor=gray,dotsize=0.2]    (4.0,6.21) {G 4}
		\dotnode[dotstyle=*,linecolor=violet,dotsize=0.2]  (4.0,6.57) {H 4}
		\dotnode[dotstyle=*,linecolor=olive,dotsize=0.2]   (4.0,7.99) {I 4}
		\dotnode[dotstyle=*,linecolor=brown,dotsize=0.2]   (4.0,8.98) {J 4}
		\dotnode[dotstyle=*,linecolor=black,dotsize=0.2]  (4.0,10.31) {L 4}

		\ncline[linecolor=blue,linestyle=dashed]{A 1}{A 2}
		\ncline[linecolor=blue,linestyle=dashed]{A 2}{A 3}
		\ncline[linecolor=blue,linestyle=dashed]{A 3}{A 4}

		\ncline[linecolor=red,linestyle=dashed]{B 0}{B 1}
		\ncline[linecolor=red,linestyle=dashed]{B 3}{B 3}

		\ncline[linecolor=cyan,linestyle=dashed]{C 0}{C 1}
		\ncline[linecolor=cyan,linestyle=dashed]{C 1}{C 2}
		\ncline[linecolor=cyan,linestyle=dashed]{C 2}{C 3}
		\ncline[linecolor=cyan,linestyle=dashed]{C 3}{C 4}

		\ncline[linecolor=magenta,linestyle=dashed]{D 0}{D 1}

		\ncline[linecolor=orange,linestyle=dashed]{E 2}{E 3}
		\ncline[linecolor=orange,linestyle=dashed]{E 3}{E 4}

		\ncline[linecolor=green,linestyle=dashed]{F 1}{F 2}

		\ncline[linecolor=gray,linestyle=dashed]{G 0}{G 0}
		\ncline[linecolor=gray,linestyle=dashed]{G 2}{G 3}
		\ncline[linecolor=gray,linestyle=dashed]{G 3}{G 4}

		\ncline[linecolor=violet,linestyle=dashed]{H 1}{H 1}
		\ncline[linecolor=violet,linestyle=dashed]{H 3}{H 4}

		\ncline[linecolor=olive,linestyle=dashed]{I 0}{I 1}
		\ncline[linecolor=olive,linestyle=dashed]{I 3}{I 4}

		\ncline[linecolor=brown,linestyle=dashed]{J 0}{J 1}
		\ncline[linecolor=brown,linestyle=dashed]{J 1}{J 2}
		\ncline[linecolor=brown,linestyle=dashed]{J 2}{J 3}
		\ncline[linecolor=brown,linestyle=dashed]{J 3}{J 4}

		\ncline[linecolor=black,linestyle=dashed]{L 1}{L 2}
		\ncline[linecolor=black,linestyle=dashed]{L 2}{L 3}
		\ncline[linecolor=black,linestyle=dashed]{L 3}{L 4}

		\section*{Join $t_0$ and $t_4$}
	\vspace{1cm}
	\psscalebox{1.0 1.0} {
	\begin{pspicture}(0,0.0)(13,12)
		\psline[linecolor=black, linewidth=0.05] (0,-1)(0,12) \uput[90](0, 12){$t_{0}$}
		\psline[linecolor=black, linewidth=0.05] (4,-1)(4,12) \uput[90](4, 12){$t_{4}$}
		\uput[180](0,0){\textcolor{blue}{$A$}}
		\uput[180](0,1){\textcolor{red}{$B$}}
		\uput[180](0,2){\textcolor{cyan}{$C$}}
		\uput[180](0,3){\textcolor{magenta}{$D$}}
		\uput[180](0,4){\textcolor{orange}{$E$}}
		\uput[180](0,5){\textcolor{green}{$F$}}
		\uput[180](0,6){\textcolor{gray}{$G$}}
		\uput[180](0,7){\textcolor{violet}{$H$}}
		\uput[180](0,8){\textcolor{olive}{$I$}}
		\uput[180](0,9){\textcolor{brown}{$J$}}
		\uput[180](0,10){\textcolor{purple}{$K$}}
		\uput[180](0,11){\textcolor{black}{$L$}}
		\dotnode[dotstyle=*,linecolor=blue,dotsize=0.2] (4.0,0.35) {A 4}
		\dotnode[dotstyle=*,linecolor=red,dotsize=0.2] (0.0,1.24) {B 0}
		\dotnode[dotstyle=*,linecolor=cyan,dotsize=0.2] (0.0,1.8) {C 0}
		\dotnode[dotstyle=*,linecolor=cyan,dotsize=0.2] (4.0,1.79) {C 4}
		\dotnode[dotstyle=*,linecolor=magenta,dotsize=0.2] (0.0,2.89) {D 0}
		\dotnode[dotstyle=*,linecolor=magenta,dotsize=0.2] (4.0,2.52) {D 4}
		\dotnode[dotstyle=*,linecolor=orange,dotsize=0.2] (0.0,3.9) {E 0}
		\dotnode[dotstyle=*,linecolor=orange,dotsize=0.2] (4.0,4.33) {E 4}
		\dotnode[dotstyle=*,linecolor=green,dotsize=0.2] (4.0,5.15) {F 4}
		\dotnode[dotstyle=*,linecolor=gray,dotsize=0.2] (0.0,5.96) {G 0}
		\dotnode[dotstyle=*,linecolor=gray,dotsize=0.2] (4.0,6.21) {G 4}
		\dotnode[dotstyle=*,linecolor=violet,dotsize=0.2] (4.0,6.57) {H 4}
		\dotnode[dotstyle=*,linecolor=olive,dotsize=0.2] (0.0,8.32) {I 0}
		\dotnode[dotstyle=*,linecolor=olive,dotsize=0.2] (4.0,7.99) {I 4}
		\dotnode[dotstyle=*,linecolor=brown,dotsize=0.2] (0.0,9.39) {J 0}
		\dotnode[dotstyle=*,linecolor=brown,dotsize=0.2] (4.0,8.98) {J 4}
		\dotnode[dotstyle=*,linecolor=purple,dotsize=0.2] (0.0,9.95) {K 0}
		\dotnode[dotstyle=*,linecolor=black,dotsize=0.2]   (4.0,10.31) {L 4}
		
		\ncline[linecolor=cyan,linestyle=dashed]{C 0}{C 4}
		\ncline[linecolor=magenta,linestyle=dashed]{D 0}{D 1}
		\ncline[linecolor=magenta,linestyle=dashed]{D 0}{D 4}
		\ncline[linecolor=orange,linestyle=dashed]{E 0}{E 4}
		\ncline[linecolor=gray,linestyle=dashed]{G 0}{G 4}
		\ncline[linecolor=olive,linestyle=dashed]{I 0}{I 4}
		\ncline[linecolor=brown,linestyle=dashed]{J 0}{J 4}
	\end{pspicture}
}
\clearpage

	\end{pspicture}
}
\clearpage
\section*{MP finding per window...}
	\vspace{1cm}
	Each trajectory is associated with just the maximal disks it touches.  MP algorithms returns sets of disks which are visited by the same trajectories.  If they happen in consecutive order, it is a flock.
	\vspace{0.5cm}
	
	Pros:
	\begin{itemize}
		\item Do not perform distance join at each timestamp.  
		\item Although still have to deal with consecutive checking, it is done just at the end of the window.
		\item It deals with subset elimination.
	\end{itemize}
	Cons:
	\begin{itemize}
		\item Overlapping disks could introduce false flocks.  It will require an additional filter at the end of the window.
	\end{itemize}
\clearpage
\section*{Some reading...}
	\begin{itemize}
		\item B. Negrevergne, A. Termier, J.-F. Méhaut, and T. Uno, “Discovering closed frequent itemsets on multicore: Parallelizing computations and optimizing memory accesses,” in High Performance Computing and Simulation (HPCS), 2010 International Conference on, 2010, pp. 521–528.
		\item M. Kirchgessner, “Mining and ranking closed itemsets from large-scale transactional datasets,” Université Grenoble Alpes, 2016.
		\item S. Cong, J. Han, and D. Padua, “Parallel mining of closed sequential patterns,” in Proceedings of the eleventh ACM SIGKDD international conference on Knowledge discovery in data mining, 2005, pp. 562–567.


	\end{itemize}

\clearpage
	