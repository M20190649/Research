\documentclass{article}
\usepackage{graphicx}
\usepackage{enumitem}
\usepackage{algorithm}
\usepackage{algpseudocode}
\usepackage{caption}
\usepackage{amsthm}
\usepackage[margin=2cm]{geometry}

\pagestyle{empty}
\setlist{nosep}
\algnewcommand\algorithmicforeach{\textbf{for each}}
\algdef{S}[FOR]{ForEach}[1]{\algorithmicforeach\ #1\ \algorithmicdo}
\algnewcommand\algorithmicswitch{\textbf{switch}}
\algnewcommand\algorithmiccase{\textbf{case}}
\algdef{SE}[SWITCH]{Switch}{EndSwitch}[1]{\algorithmicswitch\ #1\ \algorithmicdo}{\algorithmicend\ \algorithmicswitch}%
\algdef{SE}[CASE]{Case}{EndCase}[1]{\algorithmiccase\ #1}{\algorithmicend\ \algorithmiccase}%
\algtext*{EndSwitch}%
\algtext*{EndCase}%
\newtheorem*{lemma}{Lemma}

\begin{document}
    \section*{Cell record}
    Cell:
    \begin{itemize}
        \item id: Int.  Unique cell identifier.
        \item lineage: String.  The branch of the cell in the quadtree.  It provides position and depth of the cell. 
        \item envelope: Polygon.  Geometric representation of the cell.
    \end{itemize}
    
    \section*{Algorithms}
    \begin{algorithm} \caption{\textsc{getNextCellWithEdges} algorithm}
        \textbf{Require:} a quadtree with cell envelopes $\mathcal Q$ and map of cells and their edge count $\mathcal M$.
        \begin{algorithmic}[1]
        \Function{ getNextCellWithEdges }{ $\mathcal Q$, $\mathcal M$ }
            \State $\mathcal C \gets $ list of empty cells in $\mathcal M$
            \ForEach{ $emptyCell$ in $\mathcal C $ }
                \State initialize $cellList$ with $emptyCell$ 
                \State $nextCellWithEdges \gets null$
                \State $referenceCorner \gets null$
                \State $done \gets false$
                \While{ not $done$ } 
                    \State $c \gets $ last cell in $cellList$ 
                    \State $cells, corner \gets \textsc{getCellsAtCorner}(\mathcal Q, c)$ \Comment{ return 3 cells and the reference corner }
                    \ForEach{$cell$ in $cells$}
                        \State $nedges \gets$ get edge count of $cell$ in $\mathcal M$ 
                        \If{ $nedges > 0$ }
                            \State $nextCellWithEdges \gets cell$
                            \State $referenceCorner \gets corner$
                            \State $done \gets true$
                        \Else
                            \State add $cell$ to $cellList$
                        \EndIf
                    \EndFor
                \EndWhile
                \ForEach{ $cell$ in $cellList$ }
                    \State \textbf{output}($cell$, $nextCellWithEdges$, $referenceCorner$)
                    \State remove $cell$ from $\mathcal C$
                \EndFor
            \EndFor
        \EndFunction
        \end{algorithmic}
    \end{algorithm}
    
    \begin{algorithm} \caption{\textsc{getCellsAtCorner} algorithm}
        \textbf{Require:} a quadtree with cell envelopes $\mathcal Q$ and a cell $c$.
        \begin{algorithmic}[1]
        \Function{ getCellsInCorner }{ $\mathcal Q$, $c$ }
            \State $region \gets $ last character in $c.lineage$
            \Switch{ $region$ }
                \Case{ `0' }
                    \State $corner \gets$ left bottom corner of $c.envelope$
                \EndCase
                \Case{ `1' }
                    \State $corner \gets$ right bottom corner of $c.envelope$
                \EndCase
                \Case{ `2' }
                    \State $corner \gets$ left upper corner of $c.envelope$
                \EndCase
                \Case{ `3' }
                    \State $corner \gets$ right upper corner of $c.envelope$
                \EndCase
            \EndSwitch
            \State $cells \gets$ cells which intersect $corner$ in $\mathcal Q$
            \State $cells \gets cells - c$ \Comment{ Remove the current cell from the intersected cells }
            \State $cells \gets$ sort $cells$ on basis of their depth \Comment{ using $cell.lineage$ }
            \State \Return{ ($cells$, $corner$) }
        \EndFunction
        \end{algorithmic}
    \end{algorithm}
\end{document}
